\chapter{Proposed System}\label{ch:proposedsytem}

Explain your proposed system/mechanism/technique.
Figures are effective so better use them accordingly.
Vector images are more beautiful than bitmaps so it is recommended to use pdf files.\footnote{Same can be said for graphs which we will see in \cref{sec:results}.}
One caveat is that you do not want to have blank spaces surrounding your pdf file.
If you have some, you can delete them by using commands such as:

\begin{minted}[bgcolor=gray90]{bash}
% pdfcrop --margins 0 input.pdf output.pdf
\end{minted}

% Code below becomes an error for xelatex (not for platex)
% \begin{lstlisting}[language=bash,escapeinside={(*}{*)}]
% (*\colorbox{gray90}{\% pdfcrop --margins 0 input.pdf output.pdf}*)
% \end{lstlisting}

\begin{figure}[ht]
  \centering
  \includegraphics[width=0.5\textwidth]{examples/figures/square}
  \caption{A square.}\label{fig:square}
\end{figure}

Always refer to the figure in your body.
\cref{fig:square} is a square.

\begin{figure}[ht]
  \centering
  \begin{subfigure}[t]{0.45\textwidth}
  \centering
    \includegraphics[width=0.8\linewidth]{examples/figures/square}
    \caption{A square.}\label{subfig:square}
  \end{subfigure}
  \quad
  \begin{subfigure}[t]{0.45\textwidth}
  \centering
    \includegraphics[width=0.8\linewidth]{examples/figures/circle}
    \caption{A circle.}\label{subfig:circle}
  \end{subfigure}
  \caption{A square and a circle.}\label{fig:square-circle}
\end{figure}

\cref{fig:square-circle} shows a square and a circle, while \cref{subfig:circle} represents only the circle.